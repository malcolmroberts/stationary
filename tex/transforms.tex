\documentclass[a4paper]{article}
\usepackage{amsmath}
\usepackage{breqn}
\usepackage{mathdef}
\usepackage{hyperref}
\usepackage{datetime}
\usepackage{natbib}

\definecolor{heavyblue}{cmyk}{1,1,0,0.25}

\usepackage{geometry}
\geometry{a4paper,margin=0cm}


\newcommand{\thetitle}{Some figures from transforms for signal
  detection in Agent-Based Models}

\newcommand{\theauthor}{Malcolm Roberts}

\hypersetup{
  pdftitle={\thetitle},
  pdfpagemode=UseOutlines,
  citebordercolor=0 0 1,
  colorlinks=true,
  allcolors=heavyblue,
  breaklinks=true,
  pdfauthor={\theauthor},
  pdfpagetransition=Dissolve,
  bookmarks=true
}

% Specify ISO date format:
\yyyymmdddate
\renewcommand{\dateseparator}{-}

\newcommand{\no}[1]{\hiderel{#1}}


\begin{document}

\begin{center}
  \thetitle{}\\
  \today.\\
  \theauthor{}
\end{center}

\input{defrun.tex}
%\def\filename{perfectsine}

\input{def_a.tex}
%\def\aval{500}
\input{def_b.tex}
%\def\bval{500}

%\section{Introduction}

\begin{figure}[htbp]
  \begin{center}        
    \includegraphics[width=0.4\textwidth]{../data}
    \caption{Input signal from \texttt{\filename} with typical cycle
      in blue.}
  \end{center}
\end{figure}

\begin{figure}[htbp]
  \begin{center}        
    \includegraphics[width=0.4\textwidth]{../data_ac}
    \caption{Autocorrelation.}
    %\caption{Input signal from \aval{} to \bval{} with linear}
  \end{center}
\end{figure}

\begin{figure}[htbp]
  \begin{center}
    \includegraphics[width=0.4\textwidth]{../data_fac}
    \caption{FFT of autocorrelation.}
    %\caption{Input signal from \aval{} to \bval{} with linear}
  \end{center}
\end{figure}


\end{document}

